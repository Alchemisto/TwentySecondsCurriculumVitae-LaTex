\PassOptionsToPackage{dvipsnames}{xcolor}
\documentclass[letterpaper]{twentysecondcv}
%%\usepackage{xcolor}
%%\usepackage



%%%%%%%%%%%%%%%%%
%%PROFILE SIDE BAR%%
%%%%%%%%%%%%%%%%%

%%%%%%%%%%%%%%%%
%%PERSONAL INFO%%%
%%%%%%%%%%%%%%%%

\profilepic{omar.jpg} %path of profile pic
\cvname{Omar Elshal} %your name
%\cvjobtitle{MSc. Student}%your actual job position
\cvdate{25 September 1992}%date of birth
\cvaddress{Yo-Kyl\"{a} 5C 22, Turku, Finland}%address
\cvnumberphone{+358 46 9483209}%telphone number
\cvmail{oam.elshal@gmail.com}%e-mail
%\cvsite{https://linkedin.com/in/oelshal}%personal site
\cvGitHub{www.github.com/Alchemisto}
\cvLinkedin{www.linkedin.com/in/oelshal}


\begin{document}

% I have been working in SW/Embedded SW development for more than 2 years inside and outside academic environments including exposure to both commercial SW architecture such as AUTOSAR and HW educational platforms like Arduino and Raspberry Pi while using a variety of programming languages, IDEs and toolchain.
\aboutme{In the last three years, I have developed my career as a software engineer with solid hands-on experience in different phases of software development cycle, starting from specifying, designing up to integration and testing. \\
I have been involved in many projects with exposure to SW platforms like AUTOSAR and educational HW platforms like Arduino and Raspberry Pi and currently Pursuing a highly qualified technical environment that could enrich my knowledge, utilize my background and experience by introducing challenges and cutting-edge technologies.\\}%About me section

%%%%%%%%%%%%%%%%%%%%%%%%%%%%%%%%%%%%%%%%%%%%%%%%%%%%%%%%%%%%%%
%%%%%%Skill bar section, each skill must have a value between 0 an 6 (float)%%%%%%%
%%%%%%%%%%%%%%%%%%%%%%%%%%%%%%%%%%%%%%%%%%%%%%%%%%%%%%%%%%%%%%
\skills{{{\LaTeX}/4.4}, {Python/3.0}, {Embedded SW debugging/3.5}, {Java/4.5}, {Embedded C\textbackslash C++/5.2}}

%%%%%%%%%%%%%%%%%%%%%%%%%%%%%%%%%%%%%%%%%%%%%%%%%%%%%%%%%%%%%%
%%%%%%Skill text section, each skill must have a value between 0 an 6%%%%%%%%%%%%
%%%%%%%%%%%%%%%%%%%%%%%%%%%%%%%%%%%%%%%%%%%%%%%%%%%%%%%%%%%%%%
\skillstext{{Business Model development/4.0}, {Multi-cultural teamwork/4.5}, {\\C\#/3.5}, {MATLAB/3.0}, {CUDA/3.5}, {\\Microsoft tools/4.5}}


\makeprofile
%%%%%%%%%%%%%%%%%%%%
%%END PROFILE SIDE BAR%%
%%%%%%%%%%%%%%%%%%%%

%%%%%%%%%%%%%%%%%%%%
%%%%%%%%BODY%%%%%%%%
%%%%%%%%%%%%%%%%%%%%

%%%%%%%%%%%%%%%%%%%%
%%SIMPLE SECTION%%%%%%
%%%%%%%%%%%%%%%%%%%%

\section{Experience}

\begin{twenty}
  \twentyitem
    {\small{Feb-Jul} 2013}
    {Research Intern - \textbf{Valeo} \textsc{R\&D Center} }
    {\textcolor{teal}{Egypt}}
    %Automotive Industry - VIAS team
    {\small{\textcolor{MidnightBlue}{ Software Development Solutions}}\\
    \small{Worked on Ethernet stack of the AUTOSAR architecture to receive vehicle's diagnostics data using Diagnostics and I/O stacks:
    \footnotesize{
    \begin{itemize}
    \setlength\itemsep{0em}
	\item Ethernet was used for the first time within Valeo's branch through this project, as automotive communication protocols like LIN and CAN have various limitations.
    
	\item I was a part of the team who configured and developed the Ethernet (TCP/IP), Diagnostics and I/O stacks following the automotive architecture standard "AUTOSAR".

	\item In addition to that, I implemented a C\# app for displaying the diagnostics data on an Evaluation board that simulates the vehicle, then made the hardware setup needed to imitate the whole process to complete the project.
    \end{itemize}
}}}
\end{twenty}

\section{Education}

%%%%%%%%%%%%%%%%%%%%%%%%%%%%%%%%%%%%%%%%%%
%%%%%%%%%%%%%TWENTY LIST ITEMS%%%%%%%%%%%%%%
%%    Four arguments: date; title; where; description %%%%
%%%%%%%%%%%%%%%%%%%%%%%%%%%%%%%%%%%%%%%%%%
\begin{twenty}
  \twentyitem
    {2015-2017}
    {Double Masters Degree {\normalfont - EIT* Digital Master Schools}}
    {\textcolor{teal}{Belgium}}
    {Embedded Systems with a minor in Entrepreneurship and Innovation}
  \twentyitem
    {\hspace{0.2cm} 2016-2017}
    {\hspace{0.1cm} M.Sc. {\normalfont - University of Turku(UTU) / \r{A}bo Akademi University(\r{A}AU)}}
    {\textcolor{teal}{Finland}}
    {Specialization in Energy Efficient Computing}
  \twentyitem
    {\hspace{0.2cm} 2015-2016}
    {\hspace{0.1cm} M.Sc. {\normalfont - Eindhoven University of Technology (TU/e)}}
    {\textcolor{teal}{The Netherlands}}
    {Majoring in Embedded Systems}
  \twentyitem
    {2009-2014}
    {B.Sc. (Very Good) - German University in Cairo (GUC) }
    {\textcolor{teal}{Egypt}}
    {Majoring in Computer Science and Engineering. \\
    \textcolor{NavyBlue} {\footnotesize Thesis: ``Diagnostics over IP for AUTOSAR'' (Excellent)  }}
    %| Advisor: Prof. Ayman \textsc{Elnagar}
    
\end{twenty}

\section{Awards and certificates}
\begin{twentyshort}
  \twentyitemshort
    {2016}
    {\small Summer School in \textcolor{black}{Business Modelling} and \textcolor{black}{Smart Energy Systems} organized by EIT Digital and Hector School of Business at Karlsruhe Institute of Technology.}
  \twentyitemshort
    {2015}
    {\small Scholarship by \textcolor{black}{EIT Digital} Master School covering study and mobility activities.}
  \twentyitemshort
    {2013}
    {\small Qualified to the final round in \textcolor{black}{Microsoft ATLC} bachelor's project competition.}
  \twentyitemshort
    {2009}
    {\small 30\% scholarship by the \textcolor{black}{GUC} for the 5-year duration of my bachelor's degree study}

\end{twentyshort}

%the project aims to exploit Energy, Delay and Area Product for the AES encryption %algorithm using \textbf{OpenMP} API on a superscalar multi-core X86 architecture %simulated by Snipersim multi-core simulator

%%%%%%%%%%%%%%%%%%%%%%%%%%%%%%%%%%%%%%%%%%
%%%%%%%%%TWENTY LIST SHORTITEMS%%%%%%%%%%%%%%
%%% Two arguments: date; title/description %%%%%%%%%%
%%%%%%%%%%%%%%%%%%%%%%%%%%%%%%%%%%%%%%%%%%
\section{Academic Projects}
\begin{twenty}
  \twentyitem
     {2017}
    {JPEG Parallelization on Intel's Multi-core Xeon processor}
    {\textcolor{Plum}{UTU}}
    {\small{Implemented using \textbf{Cilk Plus} language, \textbf{Intel Vtune Amplifier} and \textbf{Valgrind} tools to utilize the 24-cores of a Xeon machine for processing a huge set of pictures.}}
   \twentyitem
    {2016}
    {VLIW Parallelization of AES Encryption on Intel's SiliconHive}
    {\textcolor{Plum}{TU/e}}
    {\small{The main task includes exploiting an AES encryption C application on \textbf{Intel}'s SiliconHive processor architecture through HW optimization like increasing memory footprints, or code optimizations such as loop unrolling.}}
   \twentyitem
%     {2015}
%     {GPU Parallelization for a mock Bitcoin mining Application}
%     {\textcolor{Plum}{TU/e}}
%     {\small{Implemented using \textbf{cuda} along with \textit{nvvp} profiler for further optimization aiming to mine as much coins as possible in less time exploiting GPU multi-core architecture.}}
%    \twentyitem
    {2015}
    {Cattle monitoring WSN system based on ContikiOS}
    {\textcolor{Plum}{TU/e}}
    {\small{Using wireless sensor network operating system \textbf{Contiki} to develop a cattle monitoring system done through simulations using \textit{Cooja} and different mobility models to simulate the cattle movement while evaluating network performance.}}
\end{twenty}
\footnotesize{NOTE: detailed description and demos of a number of the projects are included at LinkedIn profile}

\section{Technical Skills}
\begin{itemize}
\setlength\itemsep{-0.29em}
%Windows, Linux, OS X and various 
\item Experience with multiple \textbf{linux} distributions and other indie/open-source operating systems of whether desktop, mobile or memory-constrained platforms like ContikiOS, TinyOS and Raspbian.
\item Experience with business model development in multiple entrepreneurial events.
%\item Knowledge of AUTOSAR architecture standard (Ethernet, CAN and Diagnostics stacks)?
\item Using Agile software development methodology in multiple projects.
\item Worked with windRiver compiler, winIDEA, EB Tresos, HW/SW debuggers like JTag on evaluation boards.
\item Experience with \textbf{Eclipse}, \textbf{Microsoft Visual Studio}, \textbf{KDevelop} and multiple other IDEs
\item Using Linux software development toolchain such as \textbf{gcc}, \textbf{make} and \textbf{gdb}.
\item Worked with Arduino Mega2560, Raspberry Pi 1B, Altera board DE2-115 and Intel Galileo board
\item Using git and SVN version control software for Agile development
\end{itemize}

\section{Interests}
Embedded Platforms, Entrepreneurship, Open-Source, Reverse Code Engineering, Quantum Computing\\
Artificial Intelligence, 
%Green Energy, Smart Energy Systems\\
Reading, Gaming, Travelling

{\let\thefootnote\relax\footnotetext{* \href{http://www.masterschool.eitdigital.eu/programmes/es/}{EIT: European Institute of Innovation \& Technology Double Degree Program}}}

%%%%%%%%%%%%%%%%%%%%
%%%%%ENDBODY%%%%%%%%
%%%%%%%%%%%%%%%%%%%%

\end{document} 